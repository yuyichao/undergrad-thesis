\documentclass[10pt,fleqn]{article}
\title{Improving the techniques of creating ultracold Lithium-$7$ atoms}
\author{Yichao Yu}
\begin{document}
\maketitle
\section*{Introduction}
Ultra-cold atoms are atoms that are at a temperature really close to absolute zero, typically at the order of micro Kelvin or lower. At these low temperatures, the quantum properties of the atoms, which are usually dominated by thermal effect at room temperature ($\approx300K$), becomes important and the atoms can form new states of matter including Bose-Einstein condensate (BEC) for bosons and degenerate Fermi gas for fermions.
% \par
% With the techiques developed in the past twenty years of controlling bosonic and fermionic gas of ultracold atoms, one can use such a system of ultracold atoms to simulate and study some interesting macroscopic quantum effects including the high-temperature superconductors, the quantum magnetism, etc.
\section*{Description of Proposed Work}
The goals of this thesis are to develop and improve the techniques of trapping and cooling a ultra-cold cloud of Lithium-$7$ atoms. The Lithium-$7$ atoms used in the experiment have a small mass and therefore high tunneling amplitude, making them perfect for studying lots of macroscopic quantum effects. However, compare to other atoms used in ultra-cold atoms experiment there are also some additional problems we need to solve including the small cross section for elastic collision at low temperature.

\par
The methods that are used are mainly standard laser cooling and trapping techniques including Zeeman slower, magneto optical trap (MOT), compressed MOT, magnetic trap, radio frequency evaporation, optical trap, etc. In addition to these standard method, some non-standard special techniques are also applied in the experiment including gray-molasses and resonant light compression are also used before the evaporative cooling in order to get a sufficiently low temperature and high density to solve the small cross section problem.

\par
The thesis will mainly focus on developing and improving these special techniques used in the experiment, trying to figure out what we can get from them and makes them fit better with other parts of the experiment.

\par

This work will be completed for graduation in June, 2014 under the supervision of Professor Wolfgang Ketterle in the Physics and Center for Ultra-cold Atoms.

\end{document}
