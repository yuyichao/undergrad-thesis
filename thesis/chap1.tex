\chapter{Theory}

In this chapter, I am going to describe the theories behind our experiment. It is devided into three parts. The first section explains the theories of cold Bose gas and the Bose-Einstein condensation relavant to the experiment. The second section briefly describes some important properties of the Lithium-$7$ atom. Finally, in the third section, the cooling, trapping and state manipulation techniques used in the experiment are presented, including Zeeman slower (\ref{theory:zeeman}), MOT(\ref{theory:mot}), gray molasses (\ref{theory:gm}), dark state pumping (\ref{theory:pump}), magnetic trap (\ref{theory:mt}) and optical dipole trap (\ref{theory:odt}).

\section{What is Bose-Einstein Condensate}\label{theory:bec}

Every real particles can be classified as one of the two families according to their spins, fermions which have half integer spins and bosons which have integer spins. According to quantum field theory\cite{spin-statistics1,spin-statistics2} the many-particles wave function of identical particles must be symetric or antisymetric under particle exchange for bosons or fermions respectively. For bosonic particles, because of the symmetry of the wave function, the possibility for particles to be in the same state is greatly enhances. As a result, for boson gas at untra-low temperature, a new phase of matter is formed, known as Bose-Einstein condensate, in which almost all of the particles are condensed to the ground state.

This section, theory of BEC and observation.

\subsection{BEC in Harmonic Trap}

In the experiment, in order to do cooling and manipulation on the atom cloud, ppl often use trap, in this experiment, the BEC is greated in X-ODT (ref ...)....

\subsection{Effect of interaction}

The ... only works for non-interacting.... when adding in interaction.... change shape..... this sub section describe .... in order to measure ....

\section{Lithium-$7$ Atoms}

Fine structure\\
Hyperfine structure\\
Zeeman effect

\begin{table}
\caption{$g$-factors of Lithium-$7$}
\label{li7:g-factors}
\begin{center}
\begin{tabular}{|c|c|c|}\hline
Fine Structure & $F$ & $g$-factor \\\hline
$2^2S_{1/2}$ & $2$ & $\dfrac 12$ \\\cline{2-3}
 & $1$ & $-\dfrac 12$ \\\hline
$2^2P_{1/2}$ & $2$ & $\dfrac 16$ \\\cline{2-3}
 & $1$ & $-\dfrac 16$ \\\hline
$2^2P_{3/2}$ & $1$, $2$, $3$ & $\dfrac 23$ \\\hline
\end{tabular}
\end{center}
\end{table}

% TODO high field Zeeman

\section{Cooling and Traping Theory}
The atoms used in a cold atom experiment often come from an oven at hundreds of degrees C (500 in our experiment) with low density.\\
In order to achieve low temperature and high density required by the BEC condition. Several slowing, traping and cooling steps are usually necessary to ....\\
In this section.... talk about ..... including ....\\

\subsection{Spin Flip Zeeman Slower}\label{theory:zeeman}

The atoms comes out of the oven has ...\\
\eqar{
  v=&\sqrt{\frac{k_BT}{2\pi m}}
}

The idea of Zeeman slower is .... however, if single frequency... not more than $\Gamma=5.9\text{MHz}$

% TODO \sigma^\pm

\subsection{Magneto-Optical Trap (MOT)}\label{theory:mot}

As in most cold atom experiment, our experiment starts with loading Zeeman slowed atom into a magneto-optical trap (MOT).\\
Provides both cooling and confinement.\\
In order to understand how MOT works, we can consider a one-dimensional model\\

\subsection{Gray Molasses}\label{theory:gm}

\subsection{Dark State Pumping}\label{theory:pump}
At the end of Laser cooling, the atoms are distributed in different ground states.\\
In order to trap in MT, need to go to 2, 2 / 2, 1 which are the only trappable states at both low field and high field.\\
Choose 2, 2 because we can use dark state pumping which ...

Dark State pumping with D1 light

\subsection{Evaporation in Static Magnetic Trap}\label{theory:mt}

\subsection{Evaporation in Optical Dipole Trap}\label{theory:odt}
