\chapter{Theory}

In this chapter, I am going to describe the theories behind our experiment. It is divided into three parts. The first section explains the theories of cold Bose gas and the Bose-Einstein condensation relevant to the experiment. The second section briefly describes some important properties of the Lithium-$7$ atom. Finally, in the third section, the cooling, trapping and state manipulation techniques used in the experiment are presented, including Zeeman slower (\ref{theory:zeeman}), MOT(\ref{theory:mot}), gray molasses (\ref{theory:gm}), dark state pumping (\ref{theory:pump}), magnetic trap (\ref{theory:mt}) and optical dipole trap (\ref{theory:odt}).

\section{What is Bose-Einstein Condensate}\label{theory:bec}

Every real particles can be classified as one of the two families according to their spins, fermions which have half integer spins and bosons which have integer spins. According to quantum field theory\cite{spin-statistics1,spin-statistics2} the many-particles wave function of identical particles must be symmetric or anti-symmetric under particle exchange for bosons or fermions respectively. For bosonic particles, because of the symmetry of the wave function, the possibility for particles to be in the same state is greatly enhances. As a result, boson gas at ultra-low temperature forms a Bose-Einstein condensate (BEC), in which almost all of the particles are condensed to the lowest energy state. In the following, the relevant properties of BECs such as critical temperature and density distribution are described.

\subsection{BEC in Harmonic Trap}

Since the wave function of bosons is symmetric, multiple bosons can be in the same state. From this fact, the energy distribution can be calculated for bosons,
\eqar{
  f(\varepsilon)=&\frac{1}{\ue^{\beta(\varepsilon-\mu)} - 1}
}
Since the distribution has to be possitive for all energy states, in particular the $\varepsilon=0$ ground state, we have $\mu\geqslant0$. For all the state except the ground state, this sets an finite upper limit on the number of atoms in each state for a fixed temperature. Therefore, if the number of atoms exceeds a certain value, all the extra atoms will go into the ground state. These atoms condensed in the ground state are called the Bose Einstein condensate.\\
\\
In order to calculate the atom number in the condensate as well as the critical temperature, we can estimate the maximum atom number in the excited states (thermal atoms) with an integral,
\eqar{
  N_{th}=&\int_0^\infty\frac{g(\varepsilon)}{\ue^{\beta\varepsilon} - 1}\ud\varepsilon
}
where $g(\varepsilon)$ is the energy density of states. In our experiment, we create the BEC in a harmonic optical dipole trap (see \ref{theory:odt}) for which the energy density is,
\eqar{
  g(\varepsilon)=&\frac{\varepsilon^2}{2\hbar^3 \omega^3}\\
  N_{th}=&\frac{1}{2\hbar^3 \omega^3}\int_0^\infty\frac{\varepsilon^2}{\ue^{\beta\varepsilon} - 1}\ud\varepsilon\\
  =&\frac{1}{2\hbar^3 \omega^3\beta^3}\int_0^\infty\frac{x^2}{\ue^{x} - 1}\ud x\\
  =&\frac{k_B^3T^3}{2\hbar^3 \omega^3}\zeta(3)\Gamma(3)
}
The critical temperature of the transition, determined by $N_{th}=N$,
\eqar{
  T_C=&\frac{\hbar\omega}{k_B}\sqrt[3]{\frac{2N}{\zeta(3)\Gamma(3)}}\\
  =&0.9405\frac{\hbar\omega\sqrt[3]{N}}{k_B}
}
Condensate fraction (for large $N$),
\eqar{
  \frac{N_0}{N}=&1-\frac{N_{th}}{N}\\
  =&1-\paren{\frac{T}{T}}^3
}

\subsection{Effect of interaction}

The wavefunction of the condensed atoms is only the same with the single particle ground state wavefunction when there is no interaction between atoms. For interacting Bose gas this is no longer true. The

The ... only works for non-interacting.... when adding in interaction.... change shape..... this sub section describe .... in order to measure ....

\section{Lithium-$7$ Atoms}

Fine structure\\
Hyperfine structure\\
Zeeman effect

\begin{table}
\caption{$g$-factors of Lithium-$7$}
\label{li7:g-factors}
\begin{center}
\begin{tabular}{|c|c|c|}\hline
Fine Structure & $F$ & $g$-factor \\\hline
$2^2S_{1/2}$ & $2$ & $\dfrac 12$ \\\cline{2-3}
 & $1$ & $-\dfrac 12$ \\\hline
$2^2P_{1/2}$ & $2$ & $\dfrac 16$ \\\cline{2-3}
 & $1$ & $-\dfrac 16$ \\\hline
$2^2P_{3/2}$ & $1$, $2$, $3$ & $\dfrac 23$ \\\hline
\end{tabular}
\end{center}
\end{table}

% TODO high field Zeeman

\section{Cooling and Traping Theory}

The atoms used in an alkali atom experiment often come from an atom beam coming out of a oven kept above the melting temperature of the metal. The oven used in this experiment operates at $485^\circ\text{C}$ producing an atom beam traveling at several hundred meters per second. In order to achieve low temperature and high density, several stages of slowing, traping and cooling are implemented in this experiment which finally bring the atoms down to the Bose-Einstein condensate condition. In this section, I am going to talk about the theory behind these techniques we are using in our experiment.

\subsection{Spin Flip Zeeman Slower}\label{theory:zeeman}

The atoms comes out of the oven has an average velocity determined by the Maxwell-Boltzmann distribution,
\eqar{
  \bar v=&\sqrt{\frac{k_BT}{2\pi m}}
}
We can slow down the atoms using the recoil of photon scattering by shining resonance light to the beam. However, since the atomic transition resonance is very narrow ($\Gamma=2\pi\cdot5.9\text{MHz}$) compare to the doppler shift ($\Delta\nu=\nu v/c\approx590\text{MHz}$), the atom will soon shift out of resonance once it slows down. The Zeeman slower gets around this problem using a spacially variant magnetic field and circularly polarized $\sigma^+$ light.

% TODO \sigma^\pm

\subsection{Magneto-Optical Trap (MOT)}\label{theory:mot}

As in most cold atom experiment, our experiment starts with loading Zeeman slowed atom into a magneto-optical trap (MOT).\\
Provides both cooling and confinement.\\
In order to understand how MOT works, we can consider a one-dimensional model\\

\subsection{Gray Molasses}\label{theory:gm}

\subsection{Dark State Pumping}\label{theory:pump}
At the end of Laser cooling, the atoms are distributed in different ground states.\\
In order to trap in MT, need to go to 2, 2 / 2, 1 which are the only trappable states at both low field and high field.\\
Choose 2, 2 because we can use dark state pumping which ...

Dark State pumping with D1 light

\subsection{Evaporation in Static Magnetic Trap}\label{theory:mt}

\subsection{Evaporation in Optical Dipole Trap}\label{theory:odt}
