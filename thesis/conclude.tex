\chapter*{Conclusion}
\addcontentsline{toc}{chapter}{Conclusion}

The main goal of this work was to build an apparatus to experimentally realize a Bose-Einstein condensate which can be used as a starting point to study quantum magnetism in an optical lattice making use of the light mass and Feshbach resonance. The light mass, small elastic collision cross section and large inelastic collision loss of Lithium-$7$ makes it hard to get higher phase space density. However, as shown in the previous chapter, when the non-conventional gray molasses cooling and numerically optimized RF evaporation, we have successfully archived our goal and are able to reach quantum degeneracy within $10$ seconds including $6$ seconds of MOT loading time. We have also measured the Feshbach resonance field which is important both for evaporation in ODT and for tuning interactions in a optical lattice.\\
\\
Future work on this experiment will focus on characterizing and stabilizing our BEC and putting in the optical lattice in order to study the solid stated Hamiltonian we are interested in. Some theoretical study and intermediate steps might be necessary in order to have better understanding of our apparatus and the system we would like to study and to finallize the strategy for mapping out the quantum magnetism phase diagram.
