\chapter{Introduction}

Predicted in 1924-25 by Satyendra Nath Bose and Albert Einstein from Bose statistics, the Bose Einstein condensate is a phase of matter at ultra-cold temperature that emerges completely because of quantum effects. It was first produced in the laboratory in 1995-96 at the University of Colorado Boulder, Massachusetts Institute of Technology and Rice University using laser cooling and evaporative cooling techniques. Since then, people have been using it to study many quantum effects. Among them, one effort is to simulate complex condensed matter systems using simplfied and well controlled model systems created by loading a BEC into optical lattices.\\
\\
In our experiment, we use the Lithium-$7$ atoms to create a Bose-Einstein condensate. Because of the lightness and several Feshbach resonances, the Lithium-$7$ atom has very fast dynamics and great tunability, making it a perfect candidate for simulating and studying the phase diagrams of certain condensed matter models. The ultimate goal of the experiment is to study the anti-ferromagnetic phase in the anisotropic Heisenberg model ($XXZ$ model), and the work in this thesis focuses on getting a Bose-Einstein condensate using Lithium-$7$, which is one of the important steps before studying the system in an optical lattice.\\
\\
The presentation of this thesis is divided into two chapters. In the first chapter, I discuss the theory of our experiment, including the Bose-Einstein condensate (BEC) and the various cooling and trapping techniques we use. The second chapter describes the setup of the experiment, the alignment and optimization procedure we developed and the experimental results we have got for each step.
