\chapter*{Introduction}
\addcontentsline{toc}{chapter}{Introduction}

Predicted in 1924-25 by Satyendra Nath Bose and Albert Einstein from Bose statistics, the Bose Einstein condensate is a phase of matter at ultra-cold temperature that emerges completely because of quantum effect. It was first produced in the laboratory in 1995-96 at the University of Colorado Boulder, Massachusetts Institute of Technology and Rice University using laser cooling and evaporative cooling techniques. Since then, people have been using it to study a lot of quantum effect. Among them, one effort is to simulate complex condensed matter systems using simplfied and well controlled model systems created by loading BEC into optical lattices.\\

Our experiment, Lithium-$7$\\
Light mass, Feshbach resonance.\\
Access a bigger region in the phase diagram of the Hubbard module in order to study anti-ferromagnetism and $d$-wave super conductor.\\

% TODO

The presentation of this thesis is divided into two chapters. In the first chapter, I discuss the theory of our experiment, including the Bose-Einstein condensate (BEC) and the various cooling and trapping techniques we use. The second chapter describes the setup of the experiment, the alignment and optimization procedure we developed and the experimental result we have got for each steps.
